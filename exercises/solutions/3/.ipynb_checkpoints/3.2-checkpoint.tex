

As per question,
\begin{align}
    p=2(1-p) \label{eq:2.0.1}\\
    \implies p=2/3\label{eq:2.0.2}
    \end{align}
For a binomial distribution,
\begin {align}
    \pr{X=k}= \comb{n}{k} p^k \brak{1-p}^{n-k}\label{eq:2.0.3}
    \end{align}
For the given question,
\begin{table}[h]
\begin{center}
\begin{tabular}{|c|c|c|}
\hline
 \textbf{Variable} & $n$ & $p$\\
 \hline
 \textbf{Value} & 6 & 2/3\\
 \hline
\end{tabular}
\caption{Value of variables}
\label{Tab 1}
\end{center}
\end{table}
From \eqref{eq:2.0.3} we have,
\begin{align}
\pr{X\geq4}&=\sum_{i=4}^{6}\comb{6}{i} p^i\brak{1-p}^{6-i}
\label{eq: 2.0.4}\\
&=\frac{240}{729}+\frac{192}{729}+\frac{64}{729}
\label{eq: 2.0.5}\\
&=\frac{496}{729}
\label{eq:2.0.6}
\end{align}