In Fig. \ref{fig:1.2.131}, the sample size $S$ is the area of the rectangle given by 
\begin{align}
S=3x2=6 m^2
\end{align}
The event size is the area of the circle given by 
\begin{align}
E = \pi\brak{\frac{1}{2}}^2=\frac{\pi}{4} m^2 
\end{align}
The probabilty of the dice landing in the circle is
\begin{align}
\pr{E} = \frac{E}{S} = \frac{\pi}{24}
\label{eq:1.2.131_prob}
\end{align}
%
The python code is available in 
\begin{lstlisting}
/codes/rect.py
\end{lstlisting}
The python code generates 10,000 points uniformly within the rectangle of dimensions $3 \times 2$ and checks for the number of points within the circle of radius 0.5.  The ratio of these is close to $\frac{\pi}{24}$.  Note that each time the code is run, the ratio will change, but will still be close to $\frac{\pi}{24}$.
