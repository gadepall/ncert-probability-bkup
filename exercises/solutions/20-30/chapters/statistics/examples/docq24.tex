
	\begin{enumerate}
	\item Finding Mean \\
	\begin{multline}
	\text{Mean salary} = \\\frac{\text{Supervisor's salary} + 4\times \text{Labourer's salary}}{5}
\\
= \frac{15000+5000+5000+5000+5000}{5}
	\end{multline}
	$\therefore$ the mean salary is 7000.
	
	\item Finding Median \\
	We need to arrange the salaries in ascending order.Thus we get 5000,5000,5000,5000,15000\\
	Let N = no.of employees = 5\\
	\begin{align}
	\text{Median} &= \brak{\frac{N+1}{2}}^{th} term = 3^{rd}value
	\end{align}
	$\therefore$ the median is 5000.
	
	\item Finding Mode\\
	Mode is the highest occurring frequency of the distribution. 5000 is the most repeating salary.\\
	$\therefore$ Modal salary is 5000.
\end{enumerate}
