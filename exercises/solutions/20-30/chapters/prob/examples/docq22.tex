Let $X_i \in \cbrak{0,1}$ represent the toss of each coin, with 1 being a head  Let
\begin{align}
X = X_1 + X_2 + X_3
\end{align}
Then, 
\begin{align}
\pr{E} &= \pr{\cbrak{X = 3}+ \cbrak{X = 0} }
\\
&= \pr{X = 3}+ \pr{X = 0} 
\\
&= \comb{3}{3}\brak{\frac{1}{2}}^3+\comb{3}{0}\brak{\frac{1}{2}}^3 
\\
&= \frac{1}{4}
\end{align}
\begin{align}
\pr{F} &= \pr{X \ge 2}
\\
&= \comb{3}{2}\brak{\frac{1}{2}}^3+\comb{3}{3}\brak{\frac{1}{2}}^3 
\\
&= \frac{1}{2}
\end{align}
\begin{align}
\pr{G} &= \pr{X \le 2}
\\
&= 1 - \pr{X > 2}
\\
&=1 - \comb{3}{3}\brak{\frac{1}{2}}^3
\\
&= \frac{7}{8}
\end{align}
Now, 
\begin{align}
\pr{EF} &= \pr{\sbrak{\cbrak{X = 3}+\cbrak{X = 0}}\cbrak{X \ge 2}}
\\
&= \Pr\lbrak{\cbrak{X = 3}\cbrak{X \ge 2}}
\\
&\quad +\rbrak{\cbrak{X = 0}\cbrak{X \ge 2}}
\\
&=\pr{X = 3} = \frac{1}{8}
\label{eq:2122_EF}
\end{align}
Similarly,
\begin{align}
\pr{EG} &= \pr{\sbrak{\cbrak{X = 3}+\cbrak{X = 0}}\cbrak{X \le 2}}
\\
&= \Pr\lbrak{\cbrak{X = 3}\cbrak{X \le 2}}
\\
&\quad +\rbrak{\cbrak{X = 0}\cbrak{X \le 2}}
\\
&=\pr{X = 0} = \frac{1}{8}
\label{eq:2122_EG}
\end{align}
and
\begin{align}
\pr{FG} &= \pr{\cbrak{X \ge 2}\cbrak{X \le 2}}
\\
&= \pr{\cbrak{X = 2}}
\\
&=\comb{3}{2}\brak{\frac{1}{2}}^3 = \frac{3}{8}
\label{eq:2122_FG}
\end{align}
From the above equations we see that
\begin{align}
P\brak{E F} &= P\brak{E}P\brak{F}\\
P\brak{GF} &\neq P\brak{G}P\brak{F}\\
P\brak{E G} &\neq P\brak{E}P\brak{G}
\end{align}
Hence only the pair (E,F) are independent events. The pairs (F,G) and (G,E) are dependent events.

%The sample size = Possible number of tosses=8
%\begin{align}
%\resizebox{\columnwidth}{!}{%
%\myvec{\bmat{HHH}&\bmat{TTT}&\bmat{HHT}&\bmat{HTT}&\bmat{HTH}&\bmat{TTH}\bmat{THT}&\bmat{THH}}
%}
%\end{align}
%Favourable outcome for event E = three Heads (or) three Tails
%\begin{align}
%\myvec{\bmat{HHH}&\bmat{TTT}}
%\end{align}
%
%\begin{align}
%P\brak{E} = \frac{1}{4}
%\end{align}
%
%
%Favourable outcome for event F = atleast two Heads 
%\begin{align}
%\resizebox{\columnwidth}{!}{%
%\myvec{\bmat{HHT}&\bmat{HHH}&\bmat{HTH}&\bmat{THH}}
%}
%\end{align}
%
%\begin{align}
%P\brak{F} = \frac{1}{2}
%\end{align}
%
%Favourable outcome for event G = atmost two Heads
%\begin{align}
%\resizebox{\columnwidth}{!}{%
%\myvec{\bmat{TTT}&\bmat{HHT}&\bmat{HTT}&\bmat{HTH}&\bmat{TTH}\bmat{THT}&\bmat{THH}}
%}
%\end{align}
%
%
%
%\begin{align}
%P\brak{G} = \frac{7}{8}
%\end{align}
%
%
%Favourable outcome for event E$\cap$F
%\begin{align}
%\myvec{\bmat{HHH}}
%\end{align}
%
%\begin{align}
%P\brak{E\cap F} = \frac{1}{8}
%\end{align}
%
%
%Favourable outcome for event F$\cap$G 
%\begin{align}
%\myvec{\bmat{HHH}&\bmat{HTH}&\bmat{THH}}
%\end{align}
%
%\begin{align}
%P\brak{F\cap G} = \frac{3}{8}
%\end{align}
%
%Favourable outcome for event E$\cap$G
%\begin{align}
%\myvec{\bmat{TTT}}
%\end{align}
%
%
%
%\begin{align}
%P\brak{E\cap G} = \frac{1}{8}
%\end{align}
%
%\begin{comment}
%	\begin{lstlisting}
%	/codes/triangle/q2py
%	\end{lstlisting}
%	
%	\begin{figure}[!ht]
%	\centering
%	\includegraphics[width=\columnwidth]{/figs/triangle/q2pdf}
%	\caption{Triangle of Q125}
%	\label{fig:qtwo}	
%	\end{figure}
%	
%
%\end{comment}	
%	
%	
%\end{enumerate}
