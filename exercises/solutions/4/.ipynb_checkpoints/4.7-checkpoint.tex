Here we Define Two Random Variables X and Y. \\
\begin{itemize}
    \item X Describes whether the Ball Drawn is marked with zero or non-zero digit\\
    \item Y Describes the number of balls Drawn that are marked with non-zero digit \\
    \item P(X=0) is the Probability of Drawing any ball \\
    \item  P(X=1) is the Probability that Ball Drawn is marked with non-zero digit.\\
\end{itemize}
we know that
\begin{equation}
    P(X=0)=\frac{1}{10}
\end{equation}
where Sample space consists all 10 possibilities and among  them the Favourable Outcome is Drawing any ball.\\
similarly
\begin{equation}
    P(X=1)=1-P(X=0)=\frac{9}{10}
\end{equation}
One ball among 10 is generally marked with 0.so now Sample Space remains same but Number of Favourable Outcomes becomes 9.\\  \\
Since Trials are  Bernoulli Trials Hence the Random Variable X.
let
\begin{equation}
    q=P(X=1)
\end{equation}
\begin{equation}
    1-q=P(X=0)
\end{equation}
then from equation 2 and 3 
\begin{equation}
    q=\frac{9}{10}
\end{equation}
Probability that 4 balls are drawn and none of them is marked with zero-digit is 
\begin{equation}
    P(Y=4)=\left(\frac{9}{10}\right)^4
\end{equation}
\textbf{Generalization}\\
Above task can be Generalized using Binomial Distribution as
\begin{equation}
    P(Y=k)=\binom{n}{k}(q)^k(1-q)^{n-k}
\end{equation}
\begin{itemize}
    \item where n=4
    \item k is a  discrete variable and can take values from 0 to n
    \item for n=1000,$q=\frac{999}{1000}$
\end{itemize}
%
the probability that among 4 balls Drawn  none is marked with the digit 0 is
\begin{equation}
    P(Y=4)=\left(\frac{9}{10}\right)^4
\end{equation}