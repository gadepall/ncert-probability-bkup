Number of days in a leap year can be written as:\\
\\366 = $52\times7$ + 2\\
\\Hence a leap year has 52 weeks and an extra two days.\\
\\Define a random variable $X=\{0,1\}$ as shown in below table such that $X=0$ and $X=1$ denote the leap year has 52 and 53 Tuesdays respectively.\\
\\Let us set the number of leap years one chooses from as 4900.
\begin{align}
    \tag{5.6.1}
    \therefore n(Year) = 4900 \label{eq_(0.0.1)}
\end{align}

\begin{table}[h]
\caption{}
\centering
\begin{tabular}{|c|c|c|c|}
\hline
S.No & $X$ & 2 Extra Days & $n(X)$\\
\hline
1)  & 0 & (Sun,Mon) & $700$\\
\hline
2) & 1 & (Mon,Tue) & $700$\\
\hline
3)  & 1 & (Tue,Wed)  & $700$\\
\hline      
4)  & 0 & (Wed,Thu) & $700$ \\
\hline
5) & 0 & (Thu,Fri)  & $700$ \\
\hline
6) & 0 & (Fri,Sat) & $700$\\
\hline
7) & 0 &  (Sat,Sun) & $700$\\
\hline
\end{tabular}
\label{table}
\end{table}

 \begin{align}
  \tag{5.6.2}
  \therefore  n(X=1) = 700\times2 = 1400 \label{eq_(0.0.2)}
\end{align}
 
Probability for the occurrence of the event $X=1$ is given by: (from \eqref{eq_(0.0.1)} and \eqref{eq_(0.0.2)})
\begin{align}
    \tag{Ans}
    \therefore 
     \pr{X=1} = \frac{n(X=1)}{n(Year)} = \frac{1400}{4900} = \frac{2}{7}
\end{align}